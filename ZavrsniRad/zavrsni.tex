\documentclass[times, utf8, zavrsni]{fer}
\usepackage{booktabs}

\begin{document}

% TODO: Navedite broj rada.
\thesisnumber{55}

% TODO: Navedite naslov rada.
\title{Prepoznavanje neregistriranih vozila u prometu u Republici Hrvatskoj }

% TODO: Navedite vaše ime i prezime.
\author{Matej Krehula}

\maketitle

% Ispis stranice s napomenom o umetanju izvornika rada. Uklonite naredbu \izvornik ako želite izbaciti tu stranicu.
\izvornik

% Dodavanje zahvale ili prazne stranice. Ako ne želite dodati zahvalu, naredbu ostavite radi prazne stranice.
\zahvala{}

\tableofcontents

\chapter{Uvod}
%Uvod rada. Nakon uvoda dolaze poglavlja u kojima se obrađuje tema.
Ovaj rad se bavi izradom programske podrške za prepoznavanje registarskih oznaka vozila u prometu u stvarnom vremenu. Za postizanje tog cilja se koristi
Android aplikacija  te HTTP server.

%nesto o registarskim oznakama u rh
Registarske tablice su jedinstvene oznake za vozila koja prometuju javnim prometnicama. Standardne registarske oznake za civilna vozila u Republici Hrvatskoj sastoje se od dva slova koja označavaju grad iz kojeg vozilo dolazi a takvih kombinacija ima 34. Potom dolazi grb  Republike Hrvatske nakon čega slijede 3 illi 4 broja te na kraju 1 ili 2 slova koja su od brojeva odvojena povlakom. Individualizirane tablice su  se oblikuju tako da se sastoje od dva slova koja označavaju grad
te do osam proizvoljno kombiniranih slova i brojki sve dok su slova i brojke odvojene povlakom.  Registarske tablice su napravljene od metala bijele boje, a slova na njima su crne boje. 
%napisi  nesto o tome sto je u kojem poglavlju





\chapter{Zaključak}
Zaključak.

\bibliography{literatura}
\bibliographystyle{fer}

\begin{sazetak}
Sažetak na hrvatskom jeziku.

\kljucnerijeci{Ključne riječi, odvojene zarezima.}
\end{sazetak}

% TODO: Navedite naslov na engleskom jeziku.
\engtitle{Title}
\begin{abstract}
Abstract.

\keywords{Keywords.}
\end{abstract}

\end{document}
